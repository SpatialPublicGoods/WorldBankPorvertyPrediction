\section{Conclusions:}

The preliminary findings of our research indicate a significant improvement in income prediction accuracy by leveraging a extensive data sources. The integration of household survey data with big data sources, such as administrative records, weather patterns, and nightlight satellite imagery, has provided a more nuanced and comprehensive understanding of the variables affecting household income. Machine learning techniques, especially advanced predictive models, have successfully identified complex, non-linear relationships within the data, leading to more robust forecasting. Preliminary results suggest that environmental factors and economic activity, as inferred from weather and nightlight data, are particularly influential in predicting income changes. These insights underline the potential of using a diverse data ecosystem to inform economic policies and social welfare programs. Our research paves the way for more dynamic, data-driven decision-making processes in economic planning and poverty alleviation strategies. Further analysis and model refinement are ongoing to validate these initial outcomes and to explore the broader applicability of our approach.